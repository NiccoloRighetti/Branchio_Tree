\begin{document}

\chapter*{Abstract}
This master thesis delves into the intricate evolutionary history of Branchiopoda, a diverse group of aquatic crustaceans, through the application of advanced phylogenetic and dating methodologies. The study aims to unravel these organisms' relationships and divergence times, shedding light on their evolutionary dynamics.

The introduction provides a comprehensive overview of the diversity within Branchiopoda and introduces key concepts, including phylogenetics and tree dating techniques. The thesis outlines its objectives, primarily focused on elucidating the evolutionary relationships and temporal aspects of Branchiopoda.

Methodologically, the study employs diverse datasets, including mitochondrial, morphological, and BUSCO sequences. Phylogenetic analyses carried out using both Maximum Likelihood and Bayesian approaches, provide insights into the evolutionary history of Branchiopoda. The research also integrates tree calibration methods, encompassing least-squares dating (lsd2), Bayesian tip-dating with MrBayes, and Bayesian node calibration using MCMCtree.

The results section offers a comprehensive analysis of the findings derived from the different datasets and calibration techniques. Notably, the mitochondrial dataset analysis raises intriguing questions concerning the long branch attraction effect. The comparisons between MCMCtree and lsd2 dating methods contribute to a nuanced understanding of their respective strengths and limitations. Furthermore, individual dataset analyses provide valuable insights into the evolutionary history of Branchiopoda.

The discussion section critically examines the implications of the results. It addresses challenges posed by long branch attraction and offers a thoughtful evaluation of the performance of various dating methods. This master thesis significantly contributes to our comprehension of Branchiopoda evolution, offering a deeper understanding of their relationships and divergence times.

In conclusion, this study employs state-of-the-art phylogenetic and dating techniques to explore the evolutionary history of Branchiopoda. The results not only enhance our knowledge of this diverse group of aquatic crustaceans but also provide broader insights into the field of evolutionary biology.

\end{document}